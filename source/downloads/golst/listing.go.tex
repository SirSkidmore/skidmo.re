\documentclass[11pt]{article}
\usepackage{parskip}
\setlength{\parindent}{11pt}
\setlength{\parindent}{0cm}

\title{listing.go}

\begin{document}
\maketitle

golist is a Go utility for producing readable Go source listings
using markdown. There are two rules it uses in producing these
markdown listings:

1. lines beginning with a double slash are treated as markdown text.
2. all other lines are indented with a tab; according to markdown's
syntax, this should produce a code listing.

Currently, the only output formats supported are writing to standard out
or a markdown file.


\begin{verbatim}
package main

import (
	"bufio"
	"flag"
	"fmt"
	"io"
	"io/ioutil"
	"os"
	"path/filepath"
	"regexp"
	"time"
)

const DefaultDateFormat = "2006-01-02 15:04:05 MST"

var (
	CommentLine     = regexp.MustCompile("^\\s*//\\s*")
	DateFormat      = DefaultDateFormat
	InputFormats    map[string]SourceTransformer
	OutputFormats   map[string]OutputWriter
	OutputDirectory string
)

\end{verbatim}

A SourceTransformer converts the source code to desired form. For example,
it might convert the source to markdown, which can then be passed to a
conversion function.


\begin{verbatim}
type SourceTransformer func(string) (string, error)

\end{verbatim}

An OutputWriter takes markdown source and an output file name, and
handles its output, whether writing to a file or displaying to screen.


\begin{verbatim}
type OutputWriter func(string, string) error

func init() {
	InputFormats = make(map[string]SourceTransformer, 0)
	InputFormats["markdown"] = SourceToMarkdown
	InputFormats["tex"] = SourceToLatex

	OutputFormats = make(map[string]OutputWriter, 0)
	OutputFormats["-"] = ScreenWriter
	OutputFormats["html"] = HtmlWriter
	OutputFormats["latex"] = PandocTexWriter
	OutputFormats["md"] = MarkdownWriter
	OutputFormats["pdf"] = PdfWriter
	OutputFormats["tex"] = TexWriter
}

\end{verbatim}

SourceToMarkdown takes a file and returns a string containing the
source converted to markdown.


\begin{verbatim}
func SourceToMarkdown(filename string) (markdown string, err error) {
	file, err := os.Open(filename)
	if err != nil {
		return
	}
	defer file.Close()
	buf := bufio.NewReader(file)

	var (
		line      string
		longLine  bool
		lineBytes []byte
		isPrefix  bool
		comment   = true
	)

	markdown += "## " + filename + "\n"
	printDate := time.Now().Format(DateFormat)
	markdown += "<small>" + printDate + "</small>\n\n"

	for {
		err = nil
		lineBytes, isPrefix, err = buf.ReadLine()
		if io.EOF == err {
			err = nil
			break
		} else if err != nil {
			break
		} else if isPrefix {
			line += string(lineBytes)

			longLine = true
			continue
		} else if longLine {
			line += string(lineBytes)
			longLine = false
		} else {
			line = string(lineBytes)
		}

		if CommentLine.MatchString(line) {
			if !comment {
				markdown += "\n"
			}
			markdown += CommentLine.ReplaceAllString(line, "")
			comment = true
		} else {
\end{verbatim}

The comment flag is used to trigger a newline
before a codeblock; in some markdown
implementations, not doing this will cause the code
block to not be displayed properly.


\begin{verbatim}
			if comment {
				markdown += "  \n"
				comment = false
			}
			markdown += "\t"
			markdown += line
		}
		markdown += "\n"
	}
	return
}

func main() {
	fDateFormat := flag.String("t", DefaultDateFormat,
		"specify a format for the listing date")
	fOutputFormat := flag.String("o", "-", "output format")
	fOutputDir := flag.String("d", ".",
		"directory listings should be saved in.")
	flag.Parse()

	DateFormat = *fDateFormat
	OutputDirectory = *fOutputDir

	var transformer SourceTransformer

	outHandler, ok := OutputFormats[*fOutputFormat]
	if !ok {
		fmt.Printf("[!] %s is not a supported output format.\n",
			*fOutputFormat)
		fmt.Println("Supported formats:")
		fmt.Println("\t-        write markdown to standard output")
		fmt.Println("\thtml     produce an HTML listing")
		fmt.Println("\tlatex    produce a LaTeX listing")
		fmt.Println("\tmd       write markdown to file")
		fmt.Println("\tpdf      produce a PDF listing")
		fmt.Println("\ttex      produce a TeX listing")
		os.Exit(1)
	}

	if *fOutputFormat != "tex" {
		transformer = InputFormats["markdown"]
	} else {
		transformer = InputFormats["tex"]
	}

	for _, sourceFile := range flag.Args() {
		out, err := transformer(sourceFile)
		if err != nil {
			fmt.Fprintf(os.Stderr,
				"[!] couldn't convert %s to listing: %s\n",
				sourceFile, err.Error())
			continue
		}
		if err := outHandler(out, sourceFile); err != nil {
			fmt.Fprintf(os.Stderr,
				"[!] couldn't convert %s to listing: %s\n",
				sourceFile, err.Error())
		}
	}

}

\end{verbatim}

GetOutFile joins the output directory with the filename.


\begin{verbatim}
func GetOutFile(filename string) string {
	return filepath.Join(OutputDirectory, filename)
}

\end{verbatim}

ScreenWriter prints the markdown to standard output.


\begin{verbatim}
func ScreenWriter(markdown string, filename string) (err error) {
	_, err = fmt.Println(markdown)
	return
}

\end{verbatim}

MarkdownWriter writes the transformed listing to a file.


\begin{verbatim}
func MarkdownWriter(listing string, filename string) (err error) {
	outFile := GetOutFile(filename + ".md")
	err = ioutil.WriteFile(outFile, []byte(listing), 0644)
	return
}
\end{verbatim}

\end{document}
